% !TEX encoding = UTF-8 Unicode
% XeLaTeX can use any Mac OS X font. See the setromanfont command below.
% Input to XeLaTeX is full Unicode, so Unicode characters can be typed directly into the source.

% The next lines tell TeXShop to typeset with xelatex, and to open and save the source with Unicode encoding.

%!TEX TS-program = xelatex
%\documentclass[UTF8,a4paper,12pt]{ctexart}%需要UTF8编码

\documentclass[12pt]{article}
\usepackage[bottom=3cm]{geometry}                % See geometry.pdf to learn the layout options. There are lots.
\geometry{letterpaper}       
\usepackage{graphicx}
\usepackage{amssymb}
\usepackage{calc}
\usepackage{graphicx}
\pagestyle{empty}


%CJK Font Setup
\usepackage[PunctStyle=kaiming,AutoFakeBold=false,AutoFakeSlant=false]{xeCJK}
%\setCJKmainfont[BoldFont={Songti SC Bold},ItalicFont={KaiTi}] {SimSun}
%\setCJKsansfont[BoldFont={Heiti SC Medium}]{Heiti SC}
%\setCJKmonofont[BoldFont={Hiragino Sans GB W6}]{Hiragino Sans GB W3}

% Will Robertson's fontspec.sty can be used to simplify font choices.
% To experiment, open /Applications/Font Book to examine the fonts provided on Mac OS X,
% and change "Hoefler Text" to any of these choices.

%English Font Setup
\usepackage{fontspec,xltxtra,xunicode}
\defaultfontfeatures{Mapping=tex-text}
\setmainfont{Times New Roman}
\setCJKmainfont{STKaitiSC-Regular} %中文字体
\setsansfont[Scale=MatchLowercase,Mapping=tex-text]{STKaitiSC-Regular}
\setmonofont[Scale=MatchLowercase]{Courier New}


\author{Zu Yuan $<$\href{mailto:xxxxx@gmail.com}%
            {xxxx@gmail.com}$>$}
%\date{}                                         % Activate to display a given date or no date

\newlength{\Han}
\settowidth{\Han}{汉}
\newcommand{\spreadCJK}[2]{\makebox[#1\Han][s]{#2}}

\begin{document}
\setlength{\headsep}{0cm}

\begin{flushright}
\parbox[c]{6em}{ %
{\small \makebox[\width][c]{学校代码:\quad10246 }\par
\makebox[\width][c]{学\phantom{占位}号:\quad 22112020088}}}
\end{flushright}
\vspace{\stretch{0.5}}

\begin{figure}[htbp]
\begin{center}
\includegraphics[width=0.7\textwidth]{FudanLOGO.eps}
\end{center}
\end{figure}

\begin{figure}[htbp]
	\begin{center}
		\includegraphics[width=0.2\textwidth]{xiaohui.eps}
	\end{center}
\end{figure}

\vspace{\stretch{0.5}}
\begin{center}
{\Huge \makebox [0.45\textwidth][s]{课程论文}\par}
\vspace{\stretch{1.5}}
{\LARGE \textsf{基于钌互连的工艺发展}\par}

\vspace{\stretch{2}}
\parbox[c]{0.5\textwidth}{%
\large \setlength{\baselineskip}{1.5\baselineskip}%
\spreadCJK{7}{院系}:\quad 微电子学院\par
\spreadCJK{7}{专业}:\quad 电子信息\par
\spreadCJK{7}{姓名}:\quad 韩宙\par
\spreadCJK{7}{完成日期}:\quad 2023年6月1日}
\end{center}

\end{document}  